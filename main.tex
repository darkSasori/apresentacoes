\documentclass[aspectratio=169]{beamer}
\usepackage[utf8]{inputenc} % codificacao de caracteres
\usepackage[T1]{fontenc}    % codificacao de fontes
\usepackage[brazil]{babel}  % idioma
\usetheme{Boadilla}         % tema
\usecolortheme{orchid}      % cores
\usefonttheme[onlymath]{serif} % fonte modo matematico

\title[\sc{VIM For Dummies}]{VIM For Dummies}
\author[Lineu Felipe]{Lineu Felipe}
\institute{Socialbase} % opcional
\date{\today}

\begin{document}

\begin{frame}
    \titlepage
\end{frame}

\begin{frame}\frametitle{Por que usar o VIM?}
    \pause
    \begin{enumerate}
        \item<1-> Muito divertido \pause
        \item<2-> Modos:
            \begin{itemize}
                \item \textbf{Normal} - Atalhos e Comandos
                \item \textbf{Visual} - Selecao
                \item \textbf{Insert}
            \end{itemize}
            \pause
        \item<3-> Plugins:
            \begin{itemize}
                \item \textbf{Vundle} - Gerenciador de plugins
                \item \textbf{Fugitive} - GIT
                \item \textbf{NERDtree} - Navegador de arquivos
            \end{itemize}
            \pause
        \item<4-> Algumas IDEs tem o modo vim:
            \begin{itemize}
                \item \textbf{Sublime}
                \item \textbf{Qt Creator}
            \end{itemize}
    \end{enumerate}
\end{frame}

\begin{frame}\frametitle{Comandos}
    ':' coloca em modo comando
    \begin{enumerate}
        \item \textbf{help} - Ajuda interna do vim
        \item \textbf{w} - Salva
        \item \textbf{q} - Fecha
        \item \textbf{wq} - Salva e fecha
        \item \textbf{s} - Substituicao por regex
        \item \textbf{[0-9]} - Vai para uma linha
        \item \textbf{set number} - Númera as linhas
    \end{enumerate}
\end{frame}

\begin{frame}\frametitle{Atalhos Basicos}
    \begin{enumerate}
        \item <1-> \textbf{i, I, a, A, r, R, o, O} - Inserção
        \item <2-> \textbf{v, V} - Selecao
        \item <3-> \textbf{h, j, k, l, gg, G, \^, \$} - Navegação
        \item <4-> \textbf{w} - Palavra
        \item <5-> \textbf{y, d, c, p} - Copiar, Recortar, Recorta e coloca em modo insert (Navegação ou Repetição) e Colar
        \item <6-> \textbf{u, :redo} - Desfazer e refazer
        \item <7-> \textbf{C-e, C-y} - Page Down e Page Up
    \end{enumerate}
\end{frame}

\begin{frame}\frametitle{Unindo Atalhos}
    Exemplos:
    \begin{enumerate}
        \item <1-> \textbf{ggVG} - Seleciona todo o texto
        \item <2-> \textbf{9 + w} - Pula '9' palavras
        \item <3-> \textbf{caw} - Atalhos usados:
            \begin{itemize}
                \item \textbf{c} - Recortar e colocar em modo insert depois do comando, (Item 2 e 5)
                \item \textbf{aw} - Uma palavra. Ver manual text-object
            \end{itemize}
    \end{enumerate}
\end{frame}

\begin{frame}\frametitle{Repetindo}
    \begin{itemize}
        \item <1-> \textbf{.} - Repete o ultimo
        \item <2-> \textbf{9 [atalho]} - Repete o atalho '9' vezes
        \item <3-> \textbf{:5,10norm! [atalho]} - Executa o atalho da linha 5 até a 10
        \item <4-> \textbf{:5,10[comando]} - Executa o comando da linha 5 até a 10
    \end{itemize}
\end{frame}

\begin{frame}\frametitle{Marks}
    \begin{itemize}
        \item \textbf{m + [uma letra]} - Cria um mark
        \item \textbf{g' + [uma letra]} - Vai para o mark
        \item \textbf{:marks} - Lista os marks criados
    \end{itemize}
\end{frame}

\begin{frame}\frametitle{Macros}
    \begin{itemize}
        \item \textbf{q + [uma letra]} - Comeca a gravar o macro
        \item \textbf{q} - Para de gravar
        \item \textbf{@ + [uma letra]} - Executa o macro
        \item \textbf{@@} - Executa o ultimo macro
        \item \textbf{9 + @ + [uma letra]} - Executa '9' vezes o macro
    \end{itemize}
\end{frame}

\begin{frame}\frametitle{Onde me encontrar}
    Minhas configuracoes do vim:
    \url{https://github.com/darkSasori/vimfiles}
    \begin{itemize}
        \item \url{https://github.com/darkSasori}
        \item \url{https://twitter.com/lineufelipe}
        \item \url{https://instagram.com/lineufelipe/}
        \item \url{https://facebook.com/lineufelipe}
    \end{itemize}
\end{frame}

\begin{frame}\frametitle{Duvidas?}
    \pause
    \begin{itemize}
        \item <1-> \textbf{:help [comando ou atalho]} \pause
        \item <2-> \textbf{GOOGLE}
    \end{itemize}
\end{frame}

{%
 \usebackgroundtemplate{
  \centering
  \includegraphics[width=\paperwidth]{fim.jpg}
 }
 }
\begin{frame}
\end{frame}

\end{document}
